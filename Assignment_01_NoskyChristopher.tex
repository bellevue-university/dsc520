% Options for packages loaded elsewhere
\PassOptionsToPackage{unicode}{hyperref}
\PassOptionsToPackage{hyphens}{url}
%
\documentclass[
]{article}
\usepackage{lmodern}
\usepackage{amssymb,amsmath}
\usepackage{ifxetex,ifluatex}
\ifnum 0\ifxetex 1\fi\ifluatex 1\fi=0 % if pdftex
  \usepackage[T1]{fontenc}
  \usepackage[utf8]{inputenc}
  \usepackage{textcomp} % provide euro and other symbols
\else % if luatex or xetex
  \usepackage{unicode-math}
  \defaultfontfeatures{Scale=MatchLowercase}
  \defaultfontfeatures[\rmfamily]{Ligatures=TeX,Scale=1}
\fi
% Use upquote if available, for straight quotes in verbatim environments
\IfFileExists{upquote.sty}{\usepackage{upquote}}{}
\IfFileExists{microtype.sty}{% use microtype if available
  \usepackage[]{microtype}
  \UseMicrotypeSet[protrusion]{basicmath} % disable protrusion for tt fonts
}{}
\makeatletter
\@ifundefined{KOMAClassName}{% if non-KOMA class
  \IfFileExists{parskip.sty}{%
    \usepackage{parskip}
  }{% else
    \setlength{\parindent}{0pt}
    \setlength{\parskip}{6pt plus 2pt minus 1pt}}
}{% if KOMA class
  \KOMAoptions{parskip=half}}
\makeatother
\usepackage{xcolor}
\IfFileExists{xurl.sty}{\usepackage{xurl}}{} % add URL line breaks if available
\IfFileExists{bookmark.sty}{\usepackage{bookmark}}{\usepackage{hyperref}}
\hypersetup{
  pdftitle={Assignment\_01\_NoskyChristopher.R},
  pdfauthor={Chris},
  hidelinks,
  pdfcreator={LaTeX via pandoc}}
\urlstyle{same} % disable monospaced font for URLs
\usepackage[margin=1in]{geometry}
\usepackage{color}
\usepackage{fancyvrb}
\newcommand{\VerbBar}{|}
\newcommand{\VERB}{\Verb[commandchars=\\\{\}]}
\DefineVerbatimEnvironment{Highlighting}{Verbatim}{commandchars=\\\{\}}
% Add ',fontsize=\small' for more characters per line
\usepackage{framed}
\definecolor{shadecolor}{RGB}{248,248,248}
\newenvironment{Shaded}{\begin{snugshade}}{\end{snugshade}}
\newcommand{\AlertTok}[1]{\textcolor[rgb]{0.94,0.16,0.16}{#1}}
\newcommand{\AnnotationTok}[1]{\textcolor[rgb]{0.56,0.35,0.01}{\textbf{\textit{#1}}}}
\newcommand{\AttributeTok}[1]{\textcolor[rgb]{0.77,0.63,0.00}{#1}}
\newcommand{\BaseNTok}[1]{\textcolor[rgb]{0.00,0.00,0.81}{#1}}
\newcommand{\BuiltInTok}[1]{#1}
\newcommand{\CharTok}[1]{\textcolor[rgb]{0.31,0.60,0.02}{#1}}
\newcommand{\CommentTok}[1]{\textcolor[rgb]{0.56,0.35,0.01}{\textit{#1}}}
\newcommand{\CommentVarTok}[1]{\textcolor[rgb]{0.56,0.35,0.01}{\textbf{\textit{#1}}}}
\newcommand{\ConstantTok}[1]{\textcolor[rgb]{0.00,0.00,0.00}{#1}}
\newcommand{\ControlFlowTok}[1]{\textcolor[rgb]{0.13,0.29,0.53}{\textbf{#1}}}
\newcommand{\DataTypeTok}[1]{\textcolor[rgb]{0.13,0.29,0.53}{#1}}
\newcommand{\DecValTok}[1]{\textcolor[rgb]{0.00,0.00,0.81}{#1}}
\newcommand{\DocumentationTok}[1]{\textcolor[rgb]{0.56,0.35,0.01}{\textbf{\textit{#1}}}}
\newcommand{\ErrorTok}[1]{\textcolor[rgb]{0.64,0.00,0.00}{\textbf{#1}}}
\newcommand{\ExtensionTok}[1]{#1}
\newcommand{\FloatTok}[1]{\textcolor[rgb]{0.00,0.00,0.81}{#1}}
\newcommand{\FunctionTok}[1]{\textcolor[rgb]{0.00,0.00,0.00}{#1}}
\newcommand{\ImportTok}[1]{#1}
\newcommand{\InformationTok}[1]{\textcolor[rgb]{0.56,0.35,0.01}{\textbf{\textit{#1}}}}
\newcommand{\KeywordTok}[1]{\textcolor[rgb]{0.13,0.29,0.53}{\textbf{#1}}}
\newcommand{\NormalTok}[1]{#1}
\newcommand{\OperatorTok}[1]{\textcolor[rgb]{0.81,0.36,0.00}{\textbf{#1}}}
\newcommand{\OtherTok}[1]{\textcolor[rgb]{0.56,0.35,0.01}{#1}}
\newcommand{\PreprocessorTok}[1]{\textcolor[rgb]{0.56,0.35,0.01}{\textit{#1}}}
\newcommand{\RegionMarkerTok}[1]{#1}
\newcommand{\SpecialCharTok}[1]{\textcolor[rgb]{0.00,0.00,0.00}{#1}}
\newcommand{\SpecialStringTok}[1]{\textcolor[rgb]{0.31,0.60,0.02}{#1}}
\newcommand{\StringTok}[1]{\textcolor[rgb]{0.31,0.60,0.02}{#1}}
\newcommand{\VariableTok}[1]{\textcolor[rgb]{0.00,0.00,0.00}{#1}}
\newcommand{\VerbatimStringTok}[1]{\textcolor[rgb]{0.31,0.60,0.02}{#1}}
\newcommand{\WarningTok}[1]{\textcolor[rgb]{0.56,0.35,0.01}{\textbf{\textit{#1}}}}
\usepackage{graphicx,grffile}
\makeatletter
\def\maxwidth{\ifdim\Gin@nat@width>\linewidth\linewidth\else\Gin@nat@width\fi}
\def\maxheight{\ifdim\Gin@nat@height>\textheight\textheight\else\Gin@nat@height\fi}
\makeatother
% Scale images if necessary, so that they will not overflow the page
% margins by default, and it is still possible to overwrite the defaults
% using explicit options in \includegraphics[width, height, ...]{}
\setkeys{Gin}{width=\maxwidth,height=\maxheight,keepaspectratio}
% Set default figure placement to htbp
\makeatletter
\def\fps@figure{htbp}
\makeatother
\setlength{\emergencystretch}{3em} % prevent overfull lines
\providecommand{\tightlist}{%
  \setlength{\itemsep}{0pt}\setlength{\parskip}{0pt}}
\setcounter{secnumdepth}{-\maxdimen} % remove section numbering

\title{Assignment\_01\_NoskyChristopher.R}
\author{Chris}
\date{2020-12-06}

\begin{document}
\maketitle

\begin{Shaded}
\begin{Highlighting}[]
\CommentTok{# Assignment: ASSIGNMENT 1}
\CommentTok{# Name: Lastname, Firstname}
\CommentTok{# Date: 2010-02-14}

\CommentTok{## Create a numeric vector with the values of 3, 2, 1 using the `c()` function}
\CommentTok{## Assign the value to a variable named `num_vector`}
\CommentTok{## Print the vector}

\NormalTok{num_vector <-}\StringTok{ }\KeywordTok{c}\NormalTok{(}\DecValTok{3}\NormalTok{, }\DecValTok{2}\NormalTok{, }\DecValTok{1}\NormalTok{)}
\NormalTok{num_vector}
\end{Highlighting}
\end{Shaded}

\begin{verbatim}
## [1] 3 2 1
\end{verbatim}

\begin{Shaded}
\begin{Highlighting}[]
\CommentTok{## Create a character vector with the values of "three", "two", "one" "using the `c()` function}
\CommentTok{## Assign the value to a variable named `char_vector`}
\CommentTok{## Print the vector}

\NormalTok{char_vector <-}\StringTok{ }\KeywordTok{c}\NormalTok{(}\StringTok{'three'}\NormalTok{, }\StringTok{'two'}\NormalTok{, }\StringTok{'one'}\NormalTok{)}
\NormalTok{char_vector}
\end{Highlighting}
\end{Shaded}

\begin{verbatim}
## [1] "three" "two"   "one"
\end{verbatim}

\begin{Shaded}
\begin{Highlighting}[]
\CommentTok{## Create a vector called `week1_sleep` representing how many hours slept each night of the week}
\CommentTok{## Use the values 6.1, 8.8, 7.7, 6.4, 6.2, 6.9, 6.6}

\NormalTok{week1_sleep <-}\StringTok{ }\KeywordTok{c}\NormalTok{(}\StringTok{"Sunday"}\NormalTok{ =}\StringTok{ }\FloatTok{6.1}\NormalTok{, }\StringTok{"Monday"}\NormalTok{ =}\StringTok{ }\FloatTok{8.8}\NormalTok{, }\StringTok{"Tuesday"}\NormalTok{ =}\StringTok{ }\FloatTok{7.7}\NormalTok{, }
                 \StringTok{"Wednesday"}\NormalTok{ =}\StringTok{ }\FloatTok{6.4}\NormalTok{, }\StringTok{"Thursday"}\NormalTok{ =}\StringTok{ }\FloatTok{6.2}\NormalTok{, }\StringTok{"Friday"}\NormalTok{ =}\StringTok{ }\FloatTok{6.9}\NormalTok{,}
                 \StringTok{"Saturday"}\NormalTok{ =}\StringTok{ }\FloatTok{6.6}\NormalTok{)}
\NormalTok{week1_sleep}
\end{Highlighting}
\end{Shaded}

\begin{verbatim}
##    Sunday    Monday   Tuesday Wednesday  Thursday    Friday  Saturday 
##       6.1       8.8       7.7       6.4       6.2       6.9       6.6
\end{verbatim}

\begin{Shaded}
\begin{Highlighting}[]
\CommentTok{## Display the amount of sleep on Tuesday of week 1 by selecting the variable index}

\NormalTok{week1_sleep[}\DecValTok{3}\NormalTok{]}
\end{Highlighting}
\end{Shaded}

\begin{verbatim}
## Tuesday 
##     7.7
\end{verbatim}

\begin{Shaded}
\begin{Highlighting}[]
\CommentTok{## Create a vector called `week1_sleep_weekdays`}
\CommentTok{## Assign the weekday values using indice slicing}

\NormalTok{week1_sleep_weekdays <-}\StringTok{ }\NormalTok{week1_sleep[}\DecValTok{1}\OperatorTok{:}\DecValTok{7}\NormalTok{]}
\NormalTok{week1_sleep_weekdays}
\end{Highlighting}
\end{Shaded}

\begin{verbatim}
##    Sunday    Monday   Tuesday Wednesday  Thursday    Friday  Saturday 
##       6.1       8.8       7.7       6.4       6.2       6.9       6.6
\end{verbatim}

\begin{Shaded}
\begin{Highlighting}[]
\CommentTok{## Add the total hours slept in week one using the `sum` function}
\CommentTok{## Assign the value to variable `total_sleep_week1`}

\NormalTok{total_sleep_week1 <-}\StringTok{ }\KeywordTok{sum}\NormalTok{(week1_sleep)}
\NormalTok{total_sleep_week1}
\end{Highlighting}
\end{Shaded}

\begin{verbatim}
## [1] 48.7
\end{verbatim}

\begin{Shaded}
\begin{Highlighting}[]
\CommentTok{## Create a vector called `week2_sleep` representing how many hours slept each night of the week}
\CommentTok{## Use the values 7.1, 7.4, 7.9, 6.5, 8.1, 8.2, 8.9}
\NormalTok{week2_sleep <-}\StringTok{ }\KeywordTok{c}\NormalTok{(}\StringTok{'Sunday'}\NormalTok{ =}\StringTok{ }\FloatTok{7.1}\NormalTok{, }\StringTok{'Monday'}\NormalTok{ =}\StringTok{ }\FloatTok{7.4}\NormalTok{, }\StringTok{'Tuesday'}\NormalTok{ =}\StringTok{ }\FloatTok{7.9}\NormalTok{,}
                 \StringTok{'Wednesday'}\NormalTok{ =}\StringTok{ }\FloatTok{6.5}\NormalTok{, }\StringTok{'Thursday'}\NormalTok{ =}\StringTok{ }\FloatTok{8.1}\NormalTok{, }\StringTok{'Friday'}\NormalTok{ =}\StringTok{ }\FloatTok{8.2}\NormalTok{,}
                 \StringTok{'Saturday'}\NormalTok{ =}\StringTok{ }\FloatTok{8.9}\NormalTok{)}
\NormalTok{week2_sleep}
\end{Highlighting}
\end{Shaded}

\begin{verbatim}
##    Sunday    Monday   Tuesday Wednesday  Thursday    Friday  Saturday 
##       7.1       7.4       7.9       6.5       8.1       8.2       8.9
\end{verbatim}

\begin{Shaded}
\begin{Highlighting}[]
\CommentTok{## Add the total hours slept in week two using the sum function}
\CommentTok{## Assign the value to variable `total_sleep_week2`}
\NormalTok{total_sleep_week2 <-}\StringTok{ }\KeywordTok{sum}\NormalTok{(week2_sleep)}
\NormalTok{total_sleep_week2}
\end{Highlighting}
\end{Shaded}

\begin{verbatim}
## [1] 54.1
\end{verbatim}

\begin{Shaded}
\begin{Highlighting}[]
\CommentTok{## Determine if the total sleep in week 1 is less than week 2 by using the < operator}

\NormalTok{total_sleep_week1 }\OperatorTok{<}\StringTok{ }\NormalTok{total_sleep_week2}
\end{Highlighting}
\end{Shaded}

\begin{verbatim}
## [1] TRUE
\end{verbatim}

\begin{Shaded}
\begin{Highlighting}[]
\CommentTok{## Calculate the mean hours slept in week 1 using the `mean()` function}

\KeywordTok{mean}\NormalTok{(week1_sleep_weekdays)}
\end{Highlighting}
\end{Shaded}

\begin{verbatim}
## [1] 6.957143
\end{verbatim}

\begin{Shaded}
\begin{Highlighting}[]
\CommentTok{## Create a vector called `days` containing the days of the week.}
\CommentTok{## Start with Sunday and end with Saturday}
\NormalTok{days <-}\StringTok{ }\KeywordTok{c}\NormalTok{(}\StringTok{'Sunday'}\NormalTok{, }\StringTok{'Monday'}\NormalTok{, }\StringTok{'Tuesday'}\NormalTok{, }\StringTok{'Wednesday'}\NormalTok{, }\StringTok{'Thursday'}\NormalTok{, }\StringTok{'Friday'}\NormalTok{, }\StringTok{'Saturday'}\NormalTok{)}

\CommentTok{## Assign the names of each day to `week1_sleep` and `week2_sleep` using the `names` function and `days` vector}
\KeywordTok{names}\NormalTok{(week1_sleep) <-}\StringTok{ }\NormalTok{days}
\KeywordTok{names}\NormalTok{(week2_sleep) <-}\StringTok{ }\NormalTok{days}
\NormalTok{week1_sleep}
\end{Highlighting}
\end{Shaded}

\begin{verbatim}
##    Sunday    Monday   Tuesday Wednesday  Thursday    Friday  Saturday 
##       6.1       8.8       7.7       6.4       6.2       6.9       6.6
\end{verbatim}

\begin{Shaded}
\begin{Highlighting}[]
\CommentTok{## Display the amount of sleep on Tuesday of week 1 by selecting the variable name}
\NormalTok{week1_sleep[}\StringTok{'Tuesday'}\NormalTok{]}
\end{Highlighting}
\end{Shaded}

\begin{verbatim}
## Tuesday 
##     7.7
\end{verbatim}

\begin{Shaded}
\begin{Highlighting}[]
\CommentTok{## Create vector called weekdays from the days vector}
\NormalTok{weekdays <-}\StringTok{ }\NormalTok{days[}\DecValTok{2}\OperatorTok{:}\DecValTok{6}\NormalTok{]}
\NormalTok{weekdays}
\end{Highlighting}
\end{Shaded}

\begin{verbatim}
## [1] "Monday"    "Tuesday"   "Wednesday" "Thursday"  "Friday"
\end{verbatim}

\begin{Shaded}
\begin{Highlighting}[]
\CommentTok{## Create vector called weekends containing Sunday and Saturday}
\NormalTok{weekends <-}\StringTok{ }\NormalTok{days[}\OperatorTok{-}\NormalTok{(}\DecValTok{2}\OperatorTok{:}\DecValTok{6}\NormalTok{)]}
\NormalTok{weekends}
\end{Highlighting}
\end{Shaded}

\begin{verbatim}
## [1] "Sunday"   "Saturday"
\end{verbatim}

\begin{Shaded}
\begin{Highlighting}[]
\CommentTok{## Calculate the mean about sleep on weekdays for each week}
\CommentTok{## Assign the values to weekdays1_mean and weekdays2_mean}

\NormalTok{weekdays1_mean <-}\StringTok{ }\KeywordTok{mean}\NormalTok{(week1_sleep[weekdays])}
\NormalTok{weekdays2_mean <-}\StringTok{ }\KeywordTok{mean}\NormalTok{(week2_sleep[weekdays])}
\NormalTok{weekdays1_mean}
\end{Highlighting}
\end{Shaded}

\begin{verbatim}
## [1] 7.2
\end{verbatim}

\begin{Shaded}
\begin{Highlighting}[]
\NormalTok{weekdays2_mean}
\end{Highlighting}
\end{Shaded}

\begin{verbatim}
## [1] 7.62
\end{verbatim}

\begin{Shaded}
\begin{Highlighting}[]
\CommentTok{## Using the weekdays1_mean and weekdays2_mean variables,}
\CommentTok{## see if weekdays1_mean is greater than weekdays2_mean using the `>` operator}
\NormalTok{weekdays1_mean }\OperatorTok{>}\StringTok{ }\NormalTok{weekdays2_mean}
\end{Highlighting}
\end{Shaded}

\begin{verbatim}
## [1] FALSE
\end{verbatim}

\begin{Shaded}
\begin{Highlighting}[]
\CommentTok{## Determine how many days in week 1 had over 8 hours of sleep using the `>` operator}
\NormalTok{week1_sleep[weekdays] }\OperatorTok{>}\StringTok{ }\DecValTok{8}
\end{Highlighting}
\end{Shaded}

\begin{verbatim}
##    Monday   Tuesday Wednesday  Thursday    Friday 
##      TRUE     FALSE     FALSE     FALSE     FALSE
\end{verbatim}

\begin{Shaded}
\begin{Highlighting}[]
\CommentTok{## Create a matrix from the following three vectors}

\NormalTok{student01 <-}\StringTok{ }\KeywordTok{c}\NormalTok{(}\FloatTok{100.0}\NormalTok{, }\FloatTok{87.1}\NormalTok{)}
\NormalTok{student02 <-}\StringTok{ }\KeywordTok{c}\NormalTok{(}\FloatTok{77.2}\NormalTok{, }\FloatTok{88.9}\NormalTok{)}
\NormalTok{student03 <-}\StringTok{ }\KeywordTok{c}\NormalTok{(}\FloatTok{66.3}\NormalTok{, }\FloatTok{87.9}\NormalTok{)}

\NormalTok{students_combined <-}\StringTok{ }\KeywordTok{c}\NormalTok{(student01, student02, student03)}
\NormalTok{grades <-}\StringTok{ }\KeywordTok{matrix}\NormalTok{(}\KeywordTok{c}\NormalTok{(students_combined), }\DataTypeTok{nrow =} \DecValTok{3}\NormalTok{, }\DataTypeTok{byrow =} \OtherTok{TRUE}\NormalTok{)}

\NormalTok{grades}
\end{Highlighting}
\end{Shaded}

\begin{verbatim}
##       [,1] [,2]
## [1,] 100.0 87.1
## [2,]  77.2 88.9
## [3,]  66.3 87.9
\end{verbatim}

\begin{Shaded}
\begin{Highlighting}[]
\CommentTok{## Add a new student row with `rbind()`}
\NormalTok{student04 <-}\StringTok{ }\KeywordTok{c}\NormalTok{(}\FloatTok{95.2}\NormalTok{, }\FloatTok{94.1}\NormalTok{)}
\NormalTok{grades <-}\StringTok{ }\KeywordTok{rbind}\NormalTok{(grades, student04)}
\NormalTok{grades}
\end{Highlighting}
\end{Shaded}

\begin{verbatim}
##            [,1] [,2]
##           100.0 87.1
##            77.2 88.9
##            66.3 87.9
## student04  95.2 94.1
\end{verbatim}

\begin{Shaded}
\begin{Highlighting}[]
\CommentTok{## Add a new assignment column with `cbind()`}
\NormalTok{assignment04 <-}\StringTok{ }\KeywordTok{c}\NormalTok{(}\FloatTok{92.1}\NormalTok{, }\FloatTok{84.3}\NormalTok{, }\FloatTok{75.1}\NormalTok{, }\FloatTok{97.8}\NormalTok{)}
\NormalTok{grades <-}\StringTok{ }\KeywordTok{cbind}\NormalTok{(grades, assignment04)}
\NormalTok{grades}
\end{Highlighting}
\end{Shaded}

\begin{verbatim}
##                      assignment04
##           100.0 87.1         92.1
##            77.2 88.9         84.3
##            66.3 87.9         75.1
## student04  95.2 94.1         97.8
\end{verbatim}

\begin{Shaded}
\begin{Highlighting}[]
\CommentTok{## Add the following names to columns and rows using `rownames()` and `colnames()`}
\NormalTok{assignments <-}\StringTok{ }\KeywordTok{c}\NormalTok{(}\StringTok{"Assignment 1"}\NormalTok{, }\StringTok{"Assignment 2"}\NormalTok{, }\StringTok{"Assignment 3"}\NormalTok{)}
\NormalTok{students <-}\StringTok{ }\KeywordTok{c}\NormalTok{(}\StringTok{"Florinda Baird"}\NormalTok{, }\StringTok{"Jinny Foss"}\NormalTok{, }\StringTok{"Lou Purvis"}\NormalTok{, }\StringTok{"Nola Maloney"}\NormalTok{)}

\KeywordTok{rownames}\NormalTok{(grades) <-}\StringTok{ }\NormalTok{students}
\KeywordTok{colnames}\NormalTok{(grades) <-}\StringTok{ }\NormalTok{assignments}
\NormalTok{grades}
\end{Highlighting}
\end{Shaded}

\begin{verbatim}
##                Assignment 1 Assignment 2 Assignment 3
## Florinda Baird        100.0         87.1         92.1
## Jinny Foss             77.2         88.9         84.3
## Lou Purvis             66.3         87.9         75.1
## Nola Maloney           95.2         94.1         97.8
\end{verbatim}

\begin{Shaded}
\begin{Highlighting}[]
\CommentTok{## Total points for each assignment using `colSums()`}
\NormalTok{total_points <-}\StringTok{ }\KeywordTok{colSums}\NormalTok{(grades)}
\NormalTok{total_points}
\end{Highlighting}
\end{Shaded}

\begin{verbatim}
## Assignment 1 Assignment 2 Assignment 3 
##        338.7        358.0        349.3
\end{verbatim}

\begin{Shaded}
\begin{Highlighting}[]
\CommentTok{## Total points for each student using `rowSums()`}
\NormalTok{total_points_student <-}\StringTok{ }\KeywordTok{rowSums}\NormalTok{(grades)}
\NormalTok{total_points_student}
\end{Highlighting}
\end{Shaded}

\begin{verbatim}
## Florinda Baird     Jinny Foss     Lou Purvis   Nola Maloney 
##          279.2          250.4          229.3          287.1
\end{verbatim}

\begin{Shaded}
\begin{Highlighting}[]
\CommentTok{## Matrix with 10% and add it to grades}
\NormalTok{weighted_grades <-}\StringTok{ }\NormalTok{grades }\OperatorTok{*}\StringTok{ }\FloatTok{0.1} \OperatorTok{+}\StringTok{ }\NormalTok{grades}
\NormalTok{weighted_grades}
\end{Highlighting}
\end{Shaded}

\begin{verbatim}
##                Assignment 1 Assignment 2 Assignment 3
## Florinda Baird       110.00        95.81       101.31
## Jinny Foss            84.92        97.79        92.73
## Lou Purvis            72.93        96.69        82.61
## Nola Maloney         104.72       103.51       107.58
\end{verbatim}

\begin{Shaded}
\begin{Highlighting}[]
\CommentTok{## Create a factor of book genres using the genres_vector}
\CommentTok{## Assign the factor vector to factor_genre_vector}
\NormalTok{genres_vector <-}\StringTok{ }\KeywordTok{c}\NormalTok{(}\StringTok{"Fantasy"}\NormalTok{, }\StringTok{"Sci-Fi"}\NormalTok{, }\StringTok{"Sci-Fi"}\NormalTok{, }\StringTok{"Mystery"}\NormalTok{, }\StringTok{"Sci-Fi"}\NormalTok{, }\StringTok{"Fantasy"}\NormalTok{)}
\NormalTok{factor_genre_vector <-}\StringTok{ }\NormalTok{genres_vector}
\NormalTok{factor_genre_vector}
\end{Highlighting}
\end{Shaded}

\begin{verbatim}
## [1] "Fantasy" "Sci-Fi"  "Sci-Fi"  "Mystery" "Sci-Fi"  "Fantasy"
\end{verbatim}

\begin{Shaded}
\begin{Highlighting}[]
\CommentTok{## Use the `summary()` function to print a summary of `factor_genre_vector`}
\KeywordTok{summary}\NormalTok{(factor_genre_vector)}
\end{Highlighting}
\end{Shaded}

\begin{verbatim}
##    Length     Class      Mode 
##         6 character character
\end{verbatim}

\begin{Shaded}
\begin{Highlighting}[]
\CommentTok{## Create ordered factor of book recommendations using the recommendations_vector}
\CommentTok{## `no` is the lowest and `yes` is the highest}
\NormalTok{recommendations_vector <-}\StringTok{ }\KeywordTok{c}\NormalTok{(}\StringTok{"neutral"}\NormalTok{, }\StringTok{"no"}\NormalTok{, }\StringTok{"no"}\NormalTok{, }\StringTok{"neutral"}\NormalTok{, }\StringTok{"yes"}\NormalTok{)}
\NormalTok{factor_recommendations_vector <-}\StringTok{ }\KeywordTok{factor}\NormalTok{(}
\NormalTok{  recommendations_vector,}
  \DataTypeTok{ordered =} \OtherTok{TRUE}\NormalTok{,}
  \DataTypeTok{levels =} \KeywordTok{c}\NormalTok{(}\StringTok{'no'}\NormalTok{, }\StringTok{'neutral'}\NormalTok{, }\StringTok{'yes'}\NormalTok{)}
\NormalTok{)}
\NormalTok{factor_recommendations_vector}
\end{Highlighting}
\end{Shaded}

\begin{verbatim}
## [1] neutral no      no      neutral yes    
## Levels: no < neutral < yes
\end{verbatim}

\begin{Shaded}
\begin{Highlighting}[]
\CommentTok{## Use the `summary()` function to print a summary of `factor_recommendations_vector`}
\KeywordTok{summary}\NormalTok{(factor_recommendations_vector)}
\end{Highlighting}
\end{Shaded}

\begin{verbatim}
##      no neutral     yes 
##       2       2       1
\end{verbatim}

\begin{Shaded}
\begin{Highlighting}[]
\CommentTok{## Using the built-in `mtcars` dataset, view the first few rows using the `head()` function}
\KeywordTok{head}\NormalTok{(mtcars)}
\end{Highlighting}
\end{Shaded}

\begin{verbatim}
##                    mpg cyl disp  hp drat    wt  qsec vs am gear carb
## Mazda RX4         21.0   6  160 110 3.90 2.620 16.46  0  1    4    4
## Mazda RX4 Wag     21.0   6  160 110 3.90 2.875 17.02  0  1    4    4
## Datsun 710        22.8   4  108  93 3.85 2.320 18.61  1  1    4    1
## Hornet 4 Drive    21.4   6  258 110 3.08 3.215 19.44  1  0    3    1
## Hornet Sportabout 18.7   8  360 175 3.15 3.440 17.02  0  0    3    2
## Valiant           18.1   6  225 105 2.76 3.460 20.22  1  0    3    1
\end{verbatim}

\begin{Shaded}
\begin{Highlighting}[]
\CommentTok{## Using the built-in mtcars dataset, view the last few rows using the `tail()` function}
\KeywordTok{tail}\NormalTok{(mtcars)}
\end{Highlighting}
\end{Shaded}

\begin{verbatim}
##                 mpg cyl  disp  hp drat    wt qsec vs am gear carb
## Porsche 914-2  26.0   4 120.3  91 4.43 2.140 16.7  0  1    5    2
## Lotus Europa   30.4   4  95.1 113 3.77 1.513 16.9  1  1    5    2
## Ford Pantera L 15.8   8 351.0 264 4.22 3.170 14.5  0  1    5    4
## Ferrari Dino   19.7   6 145.0 175 3.62 2.770 15.5  0  1    5    6
## Maserati Bora  15.0   8 301.0 335 3.54 3.570 14.6  0  1    5    8
## Volvo 142E     21.4   4 121.0 109 4.11 2.780 18.6  1  1    4    2
\end{verbatim}

\begin{Shaded}
\begin{Highlighting}[]
\CommentTok{## Create a dataframe called characters_df using the following information from LOTR}
\NormalTok{name <-}\StringTok{ }\KeywordTok{c}\NormalTok{(}\StringTok{"Aragon"}\NormalTok{, }\StringTok{"Bilbo"}\NormalTok{, }\StringTok{"Frodo"}\NormalTok{, }\StringTok{"Galadriel"}\NormalTok{, }\StringTok{"Sam"}\NormalTok{, }\StringTok{"Gandalf"}\NormalTok{, }\StringTok{"Legolas"}\NormalTok{, }\StringTok{"Sauron"}\NormalTok{, }\StringTok{"Gollum"}\NormalTok{)}
\NormalTok{race <-}\StringTok{ }\KeywordTok{c}\NormalTok{(}\StringTok{"Men"}\NormalTok{, }\StringTok{"Hobbit"}\NormalTok{, }\StringTok{"Hobbit"}\NormalTok{, }\StringTok{"Elf"}\NormalTok{, }\StringTok{"Hobbit"}\NormalTok{, }\StringTok{"Maia"}\NormalTok{, }\StringTok{"Elf"}\NormalTok{, }\StringTok{"Maia"}\NormalTok{, }\StringTok{"Hobbit"}\NormalTok{)}
\NormalTok{in_fellowship <-}\StringTok{ }\KeywordTok{c}\NormalTok{(}\OtherTok{TRUE}\NormalTok{, }\OtherTok{FALSE}\NormalTok{, }\OtherTok{TRUE}\NormalTok{, }\OtherTok{FALSE}\NormalTok{, }\OtherTok{TRUE}\NormalTok{, }\OtherTok{TRUE}\NormalTok{, }\OtherTok{TRUE}\NormalTok{, }\OtherTok{FALSE}\NormalTok{, }\OtherTok{FALSE}\NormalTok{)}
\NormalTok{ring_bearer <-}\StringTok{ }\KeywordTok{c}\NormalTok{(}\OtherTok{FALSE}\NormalTok{, }\OtherTok{TRUE}\NormalTok{, }\OtherTok{TRUE}\NormalTok{, }\OtherTok{FALSE}\NormalTok{, }\OtherTok{TRUE}\NormalTok{, }\OtherTok{TRUE}\NormalTok{, }\OtherTok{FALSE}\NormalTok{, }\OtherTok{TRUE}\NormalTok{, }\OtherTok{TRUE}\NormalTok{)}
\NormalTok{age <-}\StringTok{ }\KeywordTok{c}\NormalTok{(}\DecValTok{88}\NormalTok{, }\DecValTok{129}\NormalTok{, }\DecValTok{51}\NormalTok{, }\DecValTok{7000}\NormalTok{, }\DecValTok{36}\NormalTok{, }\DecValTok{2019}\NormalTok{, }\DecValTok{2931}\NormalTok{, }\DecValTok{7052}\NormalTok{, }\DecValTok{589}\NormalTok{)}

\NormalTok{characters_df <-}\StringTok{ }\KeywordTok{data.frame}\NormalTok{(name, race, in_fellowship, ring_bearer, age)}
\NormalTok{characters_df}
\end{Highlighting}
\end{Shaded}

\begin{verbatim}
##        name   race in_fellowship ring_bearer  age
## 1    Aragon    Men          TRUE       FALSE   88
## 2     Bilbo Hobbit         FALSE        TRUE  129
## 3     Frodo Hobbit          TRUE        TRUE   51
## 4 Galadriel    Elf         FALSE       FALSE 7000
## 5       Sam Hobbit          TRUE        TRUE   36
## 6   Gandalf   Maia          TRUE        TRUE 2019
## 7   Legolas    Elf          TRUE       FALSE 2931
## 8    Sauron   Maia         FALSE        TRUE 7052
## 9    Gollum Hobbit         FALSE        TRUE  589
\end{verbatim}

\begin{Shaded}
\begin{Highlighting}[]
\CommentTok{## Sorting the characters_df by age using the order function and assign the result to the sorted_characters_df}
\NormalTok{sorted_characters_df <-}\StringTok{ }\NormalTok{characters_df[}\KeywordTok{order}\NormalTok{(age),]}
\NormalTok{sorted_characters_df}
\end{Highlighting}
\end{Shaded}

\begin{verbatim}
##        name   race in_fellowship ring_bearer  age
## 5       Sam Hobbit          TRUE        TRUE   36
## 3     Frodo Hobbit          TRUE        TRUE   51
## 1    Aragon    Men          TRUE       FALSE   88
## 2     Bilbo Hobbit         FALSE        TRUE  129
## 9    Gollum Hobbit         FALSE        TRUE  589
## 6   Gandalf   Maia          TRUE        TRUE 2019
## 7   Legolas    Elf          TRUE       FALSE 2931
## 4 Galadriel    Elf         FALSE       FALSE 7000
## 8    Sauron   Maia         FALSE        TRUE 7052
\end{verbatim}

\begin{Shaded}
\begin{Highlighting}[]
\CommentTok{## Use `head()` to output the first few rows of `sorted_characters_df`}
\KeywordTok{head}\NormalTok{(sorted_characters_df)}
\end{Highlighting}
\end{Shaded}

\begin{verbatim}
##      name   race in_fellowship ring_bearer  age
## 5     Sam Hobbit          TRUE        TRUE   36
## 3   Frodo Hobbit          TRUE        TRUE   51
## 1  Aragon    Men          TRUE       FALSE   88
## 2   Bilbo Hobbit         FALSE        TRUE  129
## 9  Gollum Hobbit         FALSE        TRUE  589
## 6 Gandalf   Maia          TRUE        TRUE 2019
\end{verbatim}

\begin{Shaded}
\begin{Highlighting}[]
\CommentTok{## Select all of the ring bearers from the dataframe and assign it to ringbearers_df}
\NormalTok{ringbearers_df <-}\StringTok{ }\NormalTok{characters_df[characters_df}\OperatorTok{$}\NormalTok{ring_bearer }\OperatorTok{==}\StringTok{ }\OtherTok{TRUE}\NormalTok{,]}
\NormalTok{ringbearers_df}
\end{Highlighting}
\end{Shaded}

\begin{verbatim}
##      name   race in_fellowship ring_bearer  age
## 2   Bilbo Hobbit         FALSE        TRUE  129
## 3   Frodo Hobbit          TRUE        TRUE   51
## 5     Sam Hobbit          TRUE        TRUE   36
## 6 Gandalf   Maia          TRUE        TRUE 2019
## 8  Sauron   Maia         FALSE        TRUE 7052
## 9  Gollum Hobbit         FALSE        TRUE  589
\end{verbatim}

\begin{Shaded}
\begin{Highlighting}[]
\CommentTok{## Use `head()` to output the first few rows of `ringbearers_df`}
\KeywordTok{head}\NormalTok{(ringbearers_df)}
\end{Highlighting}
\end{Shaded}

\begin{verbatim}
##      name   race in_fellowship ring_bearer  age
## 2   Bilbo Hobbit         FALSE        TRUE  129
## 3   Frodo Hobbit          TRUE        TRUE   51
## 5     Sam Hobbit          TRUE        TRUE   36
## 6 Gandalf   Maia          TRUE        TRUE 2019
## 8  Sauron   Maia         FALSE        TRUE 7052
## 9  Gollum Hobbit         FALSE        TRUE  589
\end{verbatim}

\end{document}
